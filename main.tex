\documentclass[12pt,a4paper]{article}
\usepackage{au_journal}

% \usepackage[danish]{babel} % danish may not be present, if not try "sudo apt install texlive-lang-european" a fmtutil -sys --all

\newcommand\mymaketitle[1]{
   \rule{\textwidth}{1.6pt}\vspace*{-\baselineskip}\vspace*{2pt}
   \rule{\textwidth}{0.4pt}
   \\   
   \huge \bf #1\\
   \vspace{-8pt}
   \rule{\textwidth}{0.4pt}\vspace*{-\baselineskip}\vspace{3.2pt}
   \rule{\textwidth}{1.6pt}
}

\selectlanguage{danish}

\begin{document}

\title{
	\mymaketitle{E2ISE\\Opgave A - Kaffeautomat}
}
\author{
   \begin{tabular}{lcc}
      \multicolumn{3}{c}{\textbf{Gruppe \#37}}\\
      \textbf{Navn} & \textbf{Studienummer} & \textbf{Retning}\\
      \toprule
      Adam Ryager Høj & 201803767 & E\\
      Sigurd Skov Jensen & 201804402 & I\\
      Rasmus Kahr& 201803491 & E\\
   \end{tabular}
}
\date{5. marts 2020}
\maketitle
\vspace{5cm}
\begin{center}
   Kontaktperson: ARH - \texttt{201803767@post.au.dk}\\
   Reviewgruppe: \#38
\end{center}

% \newpage -- skiftet til clearpage - giver pænere output og formattering.
\clearpage

\tableofcontents
\listoffigures
% \newpage
\clearpage

\section{Opgave 1: Block Definition Diagram (BDD)}

\fig{Kaffeautomat_bdd}{\textwidth}{Block Definition Diagram af kaffeautomat systemet}{H}

% \newpage
\clearpage

\section{Opgave 2: Internal Block Diagram (IBD)}

Baseret på block definition diagrammet i \figref{BDD_diagram.pdf} er følgende internal block definition diagram konstrueret som set på \figref{IBD_Diagram.pdf}.

\fig{IBD_Diagram.pdf}{\textwidth}{IBD diagram for kaffeautomat-systemet.}{H}
% \newpage
\clearpage

\section{Opgave 3: Use Case sekvensdiagram}

Baseret på den givne fully-dressed use case ``Køb Produkt'' er der opstillet sekvensdiagrammet som set på \figref{SD_Diagram.png}.

\fig{SD_Diagram.png}{\textwidth}{SD diagram for kaffeautomat-systemet.}{H}
% \newpage
\clearpage

\section{Opgave 4: State Machine Diagram (STM)}

\fig{STM_diagram.pdf}{\textwidth}{State Machine diagram for kaffeautomat-systemet}{H}

\end{document}
